%% abtex2-modelo-projeto-pesquisa.tex, v-1.9.6 laurocesar
%% Copyright 2012-2016 by abnTeX2 group at http://www.abntex.net.br/ 
%%
%% This work may be distributed and/or modified under the
%% conditions of the LaTeX Project Public License, either version 1.3
%% of this license or (at your option) any later version.
%% The latest version of this license is in
%%   http://www.latex-project.org/lppl.txt
%% and version 1.3 or later is part of all distributions of LaTeX
%% version 2005/12/01 or later.
%%
%% This work has the LPPL maintenance status `maintained'.
%% 
%% The Current Maintainer of this work is the abnTeX2 team, led
%% by Lauro César Araujo. Further information are available on 
%% http://www.abntex.net.br/
%%
%% This work consists of the files abntex2-modelo-projeto-pesquisa.tex
%% and abntex2-modelo-references.bib
%%

% ------------------------------------------------------------------------
% ------------------------------------------------------------------------
% abnTeX2: Modelo de Projeto de pesquisa em conformidade com 
% ABNT NBR 15287:2011 Informação e documentação - Projeto de pesquisa -
% Apresentação 
% ------------------------------------------------------------------------ 
% ------------------------------------------------------------------------

\documentclass[
	% -- opções da classe memoir --
	12pt,				% tamanho da fonte
	openright,			% capítulos começam em pág ímpar (insere página vazia caso preciso)
	twoside,			% para impressão em recto e verso. Oposto a oneside
	a4paper,			% tamanho do papel. 
	% -- opções da classe abntex2 --
	%chapter=TITLE,		% títulos de capítulos convertidos em letras maiúsculas
	%section=TITLE,		% títulos de seções convertidos em letras maiúsculas
	%subsection=TITLE,	% títulos de subseções convertidos em letras maiúsculas
	%subsubsection=TITLE,% títulos de subsubseções convertidos em letras maiúsculas
	% -- opções do pacote babel --
	english,			% idioma adicional para hifenização
	french,				% idioma adicional para hifenização
	spanish,			% idioma adicional para hifenização
	brazil,				% o último idioma é o principal do documento
	]{abntex2}

% ---
% PACOTES
% ---

% ---
% Pacotes fundamentais 
% ---
\usepackage{lmodern}			% Usa a fonte Latin Modern
\usepackage[T1]{fontenc}		% Selecao de codigos de fonte.
\usepackage[utf8]{inputenc}		% Codificacao do documento (conversão automática dos acentos)
\usepackage{indentfirst}		% Indenta o primeiro parágrafo de cada seção.
\usepackage{color}				% Controle das cores
\usepackage{graphicx}			% Inclusão de gráficos
\usepackage{microtype} 			% para melhorias de justificação
% ---

% ---
% Pacotes adicionais, usados apenas no âmbito do Modelo Canônico do abnteX2
% ---
\usepackage{lipsum}				% para geração de dummy text
% ---

% ---
% Pacotes de citações
% ---
\usepackage[brazilian,hyperpageref]{backref}	 % Paginas com as citações na bibl
\usepackage[alf]{abntex2cite}	% Citações padrão ABNT

% --- 
% CONFIGURAÇÕES DE PACOTES
% --- 

% ---
% Configurações do pacote backref
% Usado sem a opção hyperpageref de backref
\renewcommand{\backrefpagesname}{Citado na(s) página(s):~}
% Texto padrão antes do número das páginas
\renewcommand{\backref}{}
% Define os textos da citação
\renewcommand*{\backrefalt}[4]{
	\ifcase #1 %
		Nenhuma citação no texto.%
	\or
		Citado na página #2.%
	\else
		Citado #1 vezes nas páginas #2.%
	\fi}%
% ---

% ---
% Informações de dados para CAPA e FOLHA DE ROSTO
% ---
\titulo{ O.H.M. \\  $\Omega$ Orquestra Humano Máquina $\Omega$}
\autor{Dr. Cristiano Figueiró (IHAC/UFBA)\\ Dr. Pedro Amorim Filho (CECULT/UFRB)\\ Dr. Sólon de Albuquerque Mendes (CECULT/UFRB) \\ Msc. Guilherme Soares (CECULT/UFRB) \\ Msc. Jarbas Jacome (CAHL/UFRB) \\ Msc. Bruno Rohde}
\local{Bahia, Brasil}
\data{2016, v-0.1}
\instituicao{%
  Universidade Federal da Bahia (UFBA) 
  \par
  Universidade Federal do Recôncavo Bahiano (UFRB)
}
\tipotrabalho{Tese (Doutorado)}
% O preambulo deve conter o tipo do trabalho, o objetivo, 
% o nome da instituição e a área de concentração 
\preambulo{Projeto}
% ---

% ---
% Configurações de aparência do PDF final

% alterando o aspecto da cor azul
\definecolor{blue}{RGB}{41,5,195}

% informações do PDF
\makeatletter
\hypersetup{
     	%pagebackref=true,
		pdftitle={\@title}, 
		pdfauthor={\@author},
    	pdfsubject={\imprimirpreambulo},
	    pdfcreator={LaTeX with abnTeX2},
		pdfkeywords={abnt}{latex}{abntex}{abntex2}{projeto de pesquisa}, 
		colorlinks=true,       		% false: boxed links; true: colored links
    	linkcolor=blue,          	% color of internal links
    	citecolor=blue,        		% color of links to bibliography
    	filecolor=magenta,      		% color of file links
		urlcolor=blue,
		bookmarksdepth=4
}
\makeatother
% --- 

% --- 
% Espaçamentos entre linhas e parágrafos 
% --- 

% O tamanho do parágrafo é dado por:
\setlength{\parindent}{1.3cm}

% Controle do espaçamento entre um parágrafo e outro:
\setlength{\parskip}{0.2cm}  % tente também \onelineskip

% ---
% compila o indice
% ---
\makeindex
% ---

% ----
% Início do documento
% ----
\begin{document}

% Seleciona o idioma do documento (conforme pacotes do babel)
%\selectlanguage{english}
\selectlanguage{brazil}

% Retira espaço extra obsoleto entre as frases.
\frenchspacing 

% ----------------------------------------------------------
% ELEMENTOS PRÉ-TEXTUAIS
% ----------------------------------------------------------
% \pretextual

% ---
% Capa
% ---
%\imprimircapa
% ---

% ---
% Folha de rosto
% ---
\imprimirfolhaderosto
% ---



% ---
% inserir o sumario
% ---
\pdfbookmark[0]{\contentsname}{toc}
\tableofcontents*
\cleardoublepage
% ---


% ----------------------------------------------------------
% ELEMENTOS TEXTUAIS
% ----------------------------------------------------------
\textual

% ----------------------------------------------------------
% Introdução
% ----------------------------------------------------------
\chapter*[Resumo]{Resumo}
\addcontentsline{toc}{chapter}{Resumo}

O Grupo de pesquisa Orquestra Humano Máquina dedica-se a explorar as fronteiras entre música, tecnologia, cultura e computação ubíqua. Tendo como paradigma a inovação simultânea na tecnologia e na linguagem musical. Aplicando conhecimentos em computação móvel e computação musical, produzindo software e hardware livre através de processos colaborativos e de licenças abertas. Combinando referenciais das áreas de interatividade, conectividade, produção musical e de luteria. Situando a reflexão dos processos dentro da área da Sonologia\footnote{“O termo vem sendo adotado por pesquisadores e instituições no Brasil para fazer referência a um campo híbrido de pesquisas musicais em que o som serve como elemento catalisador. Entre as áreas relacionadas estão as práticas eletroacústicas, as aplicações de novas tecnologias à produção e análise musical, a acústica e a psicoacústica, as musicologias interessadas em explorar aspectos estéticos e técnicos do som no contexto musical e os processos de criação multimidiáticos entre outras.”}. 

Dentre as atividades do grupo estão: o desenvolvimento de software e hardware para pesquisa em música, suportes para novas práticas musicais, novos paradigmas idiomáticos para instrumentos tradicionais; gerar possibilidades de criação musical colaborativa destas linguagens desenvolvidas; produção de material didático para dar subsídio ao trabalho com este conteúdo no ensino da graduação e pós-graduação.

Os projetos desenvolvidos no grupo Orquestra Mecatrônica se apoiam dentro das propostas abaixo:


\begin{enumerate}

\item Exploração das Interseções entre música e tecnologia.
\item Desenvolvimento de tecnologias livres e abertas através de licenciamentos abertos.
\item Aproximações entre a Universidade, a comunidade e as redes de criação musical.
\item Reflexão sobre a criação musical contemporânea, modos de criação, execução e escuta através de comunicações, seminários e publicação de artigos e livros.
\item Colaboração entre grupos interdisciplinares das áreas de computação, física acústica, design, música e artes visuais.
\item Atuação em espaços de livre circulação do conhecimento e prática em artes e ciências.




\end{enumerate}





% ----------------------------------------------------------
% Capitulo de textual  
% ----------------------------------------------------------
\chapter{Projeto}

% ----------------------------------------------------------

O Grupo tem uma atuação interdisciplinar ao redor da pesquisa em música e tecnologia. A urgência da criação e as demandas por novas músicas e novas maneiras de se relacionar com o som exigem a emergência de técnicas, metodologias e pactos pela criação de novos sistemas musicais. 

Amparados de um lado pelas teorias da expansão da linguagem musical a partir da problematização da escuta, dos seus contextos originais e contingências tecnológicas dos quais emergem novas identidades sonoras. Por outro lado na bibliografia técnica da computação musical, computação física e design sonoro. 

Buscamos apoiar nossa prática de pesquisa unindo a criação musical inovadora com o desenvolvimento de tecnologias livres mantendo um foco no aspecto pedagógico gerado a partir do conhecimento produzido. 

O projeto está fundamentado em 5 linhas de pesquisa:

\section{Criação musical}

Estudos sobre processos, técnicas e estratégias de composição no campo expandido da arte sonora. Abrangendo a interação da música com outras linguagens e focando em processos artísticos que problematizem criação, execução e recepção musical.

\section{Tecnologias livres}

Projetos voltados para a construção de software e hardware aplicados à composição, musicologia, performance, educação musical e design sonoro.

\section{Sonologia}

Reflexões sobre  os meios de criação e recepção ligados às artes sonoras.

“O termo [sonologia] vem sendo adotado por pesquisadores e instituições no Brasil para fazer referência a um campo híbrido de pesquisas musicais em que o som serve como elemento catalisador. Entre as áreas relacionadas estão as práticas eletroacústicas, as aplicações de novas tecnologias à produção e análise musical, a acústica e a psicoacústica, as musicologias interessadas em explorar aspectos estéticos e técnicos do som no contexto musical e os processos de criação multimidiáticos entre outras.”

Produção textual em análise e crítica dos fenômenos da arte sonora, tendo como base conceitual por um lado estudos da etnomusicologia e estudos culturais e por outro da crítica a tecnologia e do modo de existência dos objetos técnicos \cite{simondon2008modo} . Podendo se utilizar  também dos conceitos ao redor da linguagem da música eletroacústica \cite{emmerson1986language} e da musicalidade da máquina\cite{rowe2004machine}

\section{Tecnologias móveis}

Convergência de protocolos, conectividade, criação e experimentação sonora em rede, mobilidade, música locativa.

A experiência de fazer "música com dispositivos móveis"\cite{brinkmann2012making} usando áudio e interferência sonora do ambiente pode provocar alterações na percepção de tempo e espaço do usuário. Essas alterações tem sido descritas como positivas por músicos e artistas, pelo fato de possibilitar novos gestos criativo/musicais não antes pensados. O fato de alguns aplicativos serem reativos ao som do ambiente acaba criando uma trilha sonora específica de cada momento e lugar, possibilitando uma re-significação de trajetos e novas possibilidades de relação entre tempo e espaço.

\section{Prática de laboratório de acústica e computação aplicados em luteria}

Desenvolvimento de novas técnicas e metodologias de ensino coletivo da música que compreenda a dimensão física da criação do som: propriedades acústicas, eletrônicas e digitai

\section{Publicações}

O resultado das pesquisas se concretizam em artigos, composições musicais em diversos suportes e programas de computador e aplicativos para dispositivos móveis. Além do uso e distribuição gratuita dos aplicativos, é sempre possível o acesso ao código-fonte e o estudo e aprofundamento do processo de desenvolvimento. A documentação é apresentada no formato de textos técnicos (tutoriais), cobrindo a parte de instalação, uso e programação dos programas, acompanhados de vídeos demonstrativos. Um conceito norteador na pesquisa é a busca por uma  mudança de foco no uso das tecnologias enquanto ferramenta de expressividade artística e conhecimento ao invés de objeto de consumo.

\section{Interdisciplinaridade e metodologias de trabalho coletivo}

O desenvolvimento é conduzido por uma equipe de desenvolvedores/artistas com perfil interdisciplinar e experiência na área de criação musical e programação. São realizadas chamadas para bolsistas e participantes que acompanham de perto o desenvolvimento participando das fases de documentação, divulgação, avaliação e testes dos protótipos. O trabalho de pesquisa tem etapas presenciais coletivas e outras de desenvolvimento individual. Os encontros presenciais são feitos na UFBA e na UFRB, usando equipamentos pessoais e da universidade. Todo resultado é publicado abertamente sob a licença GPL, em um site amplamente divulgado nas redes sociais para a continuidade da pesquisa e referência do processo.

\chapter{Justificativa}

Continuidade de blablabla musica movel, ctlca, arrastão visual, interfaces orquestra organismo.


Esse grupo se apresenta como o primeiro a se Sonologia computação musical na UFBA UFRB. A pesquisa desenvolvida no grupo tem um forte impacto na área da pesquisa em música, atingindo subáreas como composição, musicologia, performance e educação musical. Além de influenciar também áreas correlatas como Luteria, engenharia acústica, engenharia elétrica, design de interfaces, cinema e computação. 

Ambiente contexto: UFBA , UFRB, A dinâmica do grupo contempla também uma troca entre pesquisadores docentes da UFRB e UFBA que possibilita e facilita um intercâmbio entre a região do Recôncavo e Salvador, além de contribuir dentro do contexto de linhas de pesquisa dos Bacharelados Interdisciplinares de Artes, Ciência e Tecnologia (UFBA-IHAC), Cultura, Linguagens e Tecnologias Aplicadas (UFRB-CECULT), Licenciaturas de Música e Artes (UFRB-CECULT) e Artes e Humanidades (UFRB-CAHL).

A criação da área de concentração em Artes e Tecnologias contemporâneas nos Bacharelados Interdisciplinares da UFBA (IHAC/UFBA) e o Bacharelado Interdisciplinar em Cultura, Linguagens e Tecnologias Aplicadas (CECULT/UFRB) disponibilizam um ambiente de ensino e pesquisa adequado para projetos dessa natureza. Assim, as imersões e encontros presenciais podem se valer da estrutura disponível como salas, estúdios e equipamentos da UFBA e UFRB . O público-alvo do Grupo de pesquisa são estudantes de graduação e pós-graduação dos cursos já citados além de outros cursos afins como música, artes visuais, ciências da computação, engenharia elétrica e de computação, cinema, dança, entre outros.

Os celulares e tablets modernos tem poder de processamento compatível com a criação de um estúdio portátil de produção musical, portanto é necessária a pesquisa com documentação aberta sobre programação e usabilidade de dispositivos que as pessoas já possuem e usam no dia a dia.

Entendemos que aplicativos e programas são formas de conhecimento e artigos culturais, antes de virarem produtos de consumo. De certa maneira, a relação da maioria da população com as tecnologias móveis ainda é de consumo passivo. Nessa pesquisa serão oferecidos caminhos de apropriação tecnológica de dispositivos móveis, tanto em relação a criação musical, quanto em relação as técnicas de programação para esses aparelhos.



\chapter{Metodologias}


Live coding
github
hacklab
bancadas
análise musical

\chapter{Objetivos}

% ---

\cite{brinkmann2011embedding}



% ----------------------------------------------------------
% ELEMENTOS PÓS-TEXTUAIS
% ----------------------------------------------------------
%\postextual

% ----------------------------------------------------------
% Referências bibliográficas
% ----------------------------------------------------------
\bibliography{orquestra}

% ----------------------------------------------------------
% Glossário
% ----------------------------------------------------------
%
% Consulte o manual da classe abntex2 para orientações sobre o glossário.
%
%\glossary

% ----------------------------------------------------------
% Apêndices
% ----------------------------------------------------------

% ---
% Inicia os apêndices
% ---
%\begin{apendicesenv}

% Imprime uma página indicando o início dos apêndices
%\partapendices

% ----------------------------------------------------------
%\chapter{Quisque libero justo}
% ----------------------------------------------------------

%\lipsum[50]

% ----------------------------------------------------------
%\chapter{Nullam elementum urna vel imperdiet sodales elit ipsum pharetra ligula
%ac pretium ante justo a nulla curabitur tristique arcu eu metus}
% ----------------------------------------------------------
%\lipsum[55-57]

%\end{apendicesenv}
% ---


% ----------------------------------------------------------
% Anexos
% ----------------------------------------------------------

% ---
% Inicia os anexos
% ---
%\begin{anexosenv}

% Imprime uma página indicando o início dos anexos
%\partanexos

% ---
%\chapter{Morbi ultrices rutrum lorem.}
% ---
%\lipsum[30]

% ---
%\chapter{Cras non urna sed feugiat cum sociis natoque penatibus et magnis dis
%parturient montes nascetur ridiculus mus}
% ---

%\lipsum[31]

% ---
%\chapter{Fusce facilisis lacinia dui}
% ---

%\lipsum[32]

%\end{anexosenv}

%---------------------------------------------------------------------
% INDICE REMISSIVO
%---------------------------------------------------------------------

\phantompart

\printindex


\end{document}
